\documentclass[11pt,a4paper]{article}
\usepackage[margin=1in]{geometry}
\usepackage[utf8]{inputenc}
\usepackage[T1]{fontenc}
\usepackage{hyperref}
\pagestyle{empty}

\begin{document}

\begin{flushright}
Lars R\"onnb\"ack\\
Stockholm University\\
\href{mailto:lars@uptochange.com}{lars@uptochange.com}\\[6pt]
\today
\end{flushright}

\bigskip

\noindent
The Editors\\
\textit{Journal of Mathematical Physics}\\
AIP Publishing

\bigskip

\noindent
Dear Editors,

\medskip

I am pleased to submit the manuscript

\begin{center}
\textbf{Phase-Space Seams: Invariant Volume Corrections and Universal Positivity\\
at All Even Jet Orders}
\end{center}

\noindent
for consideration for publication in the \textit{Journal of Mathematical Physics}.

\medskip

\noindent\textbf{Summary of results.}
The paper introduces \emph{phase-space seams} --- real-valued scalar fields on
$\mathbb{R}^{2d}$ whose sublevel sets serve as geometric proxies for phase-space
distributions.
The central result is a coordinate-free first-variation formula: to leading order, the
Lebesgue volume of a sublevel set $\{s \le \varepsilon\}$ is corrected by the \emph{double
trace} (the unique $O(n)$-scalar) of the quartic jet tensor, with all other irreducible
components averaging to zero on the sphere.
A universality theorem then shows that normalisability of the model density
$\exp(-s_{2k})$ forces the $k$-fold trace of the $2k$-jet tensor to be strictly positive
at every even order, so non-Gaussianity always \emph{reduces} sublevel-set volumes to
first order.
The argument is a single measure-theoretic step --- spherical averaging of a non-negative
homogeneous polynomial --- which subsumes the quartic AM-GM proof as a special case and
requires no case analysis.
Explicit formulas are given for the quartic and sextic jets in $d=1$, the multi-mode
extension to arbitrary $d$, and first-order corrections to the true second moments of
the model density.

\medskip

\noindent\textbf{Significance and scope.}
The results lie at the interface of Hessian (information) geometry, harmonic analysis on
spheres, and classical phase-space theory.
The invariant first-variation formula and the universality theorem are new, self-contained
mathematical results proved by elementary means; the paper requires no background in
quantum mechanics, though the motivation and language are drawn from phase-space
representations of quantum states.
The \textit{Journal of Mathematical Physics} is the natural venue given the paper's focus
on rigorous geometric and measure-theoretic results in a physics-motivated setting.

\medskip

\noindent\textbf{Prior publication.}
This manuscript has not been published elsewhere and is not under consideration at
another journal.
There is no related work under review or in press by the author that overlaps
significantly with the present submission.

\medskip

\noindent\textbf{Suggested referees.}
The paper touches on Hessian geometry, information geometry, and phase-space analysis.
Potential reviewers with relevant expertise include researchers working on Hessian
manifold theory, harmonic analysis on symmetric spaces, or quantum phase-space
distributions.
I have no conflicts with any potential referees.

\bigskip

\noindent
I confirm that I have no conflicts of interest to disclose and that all results are
original.
I look forward to the editors' and referees' evaluation.

\bigskip

\noindent
Yours sincerely,

\bigskip
\bigskip

\noindent
Lars R\"onnb\"ack

\end{document}
