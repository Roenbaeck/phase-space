\documentclass[11pt,a4paper]{article}
\usepackage[margin=1in]{geometry}
\usepackage{amsmath,amssymb,amsfonts,amsthm}
\usepackage{graphicx}
\usepackage{hyperref}
\usepackage{tikz}
\usetikzlibrary{shapes.geometric}

\newtheorem{definition}{Definition}[section]
\newtheorem{proposition}{Proposition}[section]
\newtheorem{theorem}{Theorem}[section]
\newtheorem{corollary}{Corollary}[section]
\newtheorem{remark}{Remark}[section]
\newtheorem{lemma}{Lemma}[section]
\newtheorem{assumption}{Assumption}[section]

\title{Phase-Space Seams: Hessian Geometry and the Invariant First Variation of Quartic Sublevel Sets}
\author{Lars R\"onnb\"ack \\
        Stockholm University \\
        \texttt{lars@uptochange.com}}
\date{\today}

\begin{document}

\maketitle

\begin{abstract}
We introduce \emph{phase-space seams}, a scalar-first framework in which a single
real-valued scalar field \(s\) on \(\mathbb{R}^{2d}\) generates local geometric data (via a
Hessian rule) and a family of sublevel sets.

Our main general result is an invariant first-variation formula in \(\mathbb{R}^n\): to first order,
the volume of small sublevel sets \(\{s\le \varepsilon\}\) depends on the quartic jet only through the
unique \(O(n)\)-scalar contraction (the double trace) of the 4-jet tensor. For the quartic sublevel set
\(\{s_4\le 1/2\}\) we also show a universality statement: normalisability of the model density
\(\exp(-s_4)\) forces positivity of the corresponding first-order correction coefficient.

In particular, for the quadratic seam
\[
s(x,p) = \frac{x^2}{2\sigma_x^2} + \frac{p^2}{2\sigma_p^2},
\]
with positive width parameters \(\sigma_x,\sigma_p\). Its sublevel set \(\{s\le 1/2\}\) is
an ellipsoid whose Lebesgue volume is
\(\frac{\pi^d}{d!}\prod_i\sigma_{x_i}\sigma_{p_i}\).

We then study \emph{quartic seams} obtained by retaining the full fourth-order Taylor jet,
including the cross-term \(x^2p^2\). A coordinate-invariant 4-jet contraction must
include all fourth-derivative components; the cross-coupling contributes with a
coefficient \(1/8\), distinct from the diagonal coefficient \(3/16\), reflecting universal
angular integrals.

Finally, for the model density \(\rho\propto\exp(-s_4)\) we compute true second moments to
first order in the quartic strength \(\lambda\), making explicit that the moments of
\(\rho\) generally differ from the jet parameters \(\sigma_x,\sigma_p\) in non-Gaussian
models.
\end{abstract}

%------------------------------------------------------------------
\section{Introduction}
%------------------------------------------------------------------

Geometry is often presented as a structure postulated \emph{a priori}. Here we adopt the
complementary viewpoint: begin with a base space and a real-valued scalar field (the
\emph{seam}) and let an explicit local rule generate the geometric data. In this note we
apply this scalar-first philosophy to classical phase space \(\mathbb{R}^{2d}\) equipped
with the standard symplectic coordinates \((x,p)\).

From the viewpoint of information geometry, the central construction is Hessian: when a
seam is interpreted as a negative log-density, its Hessian is the observed-information
matrix, and its expectation recovers the Fisher information metric under standard
regularity hypotheses \cite{Rao1945,AmariNagaoka2000}.

More generally, in the standard affine coordinates on \(\mathbb{R}^{2d}\), the Hessian rule produces a
Hessian metric in the sense of Hessian manifold theory; see, e.g., \cite{Shima2007}.

We focus first on the simplest non-trivial seams --- quadratic forms --- and compute the
Lebesgue volume of their sublevel sets explicitly. We then extend this to the full
quartic seam and obtain a first-order area correction, correcting an incompleteness in
the diagonal-only treatment: the cross-coupling term \(x^2 p^2\) must be included in any
coordinate-invariant formulation, and it enters with a universal weight dictated by
angular integrals. We also show that normalisability of the model density \(\exp(-s_4)\)
forces positivity of the first-order correction coefficient.

The construction is self-contained and uses only elementary multivariable calculus and
the AM-GM inequality. We outline immediate computational applications and sketch the
natural multi-mode generalisation in dimension \(d > 1\).

%------------------------------------------------------------------
\section{Background and Scope}
%------------------------------------------------------------------

The formalism in this note is geometric: it starts from a scalar field \(s\) on phase
space and derives objects such as a Hessian metric and level-set geometry. Connections
to quantum mechanics can be made by choosing a phase-space representation (e.g. a Husimi
\(Q\)-function) and identifying \(s\) with a log-density; we use this as motivation and
do not assume that arbitrary seam level sets are constrained directly by the canonical
commutation relation.

\paragraph{Phase-space representations.}
Quantum states admit phase-space descriptions such as the Wigner function, which can be
negative and therefore cannot globally be written as \(\exp(-s)\) with real \(s\)
\cite{Wigner1932,Hillery1984}. Positive phase-space distributions are provided by the Husimi
\(Q\)-function (a Gaussian smoothing of the Wigner function) \cite{Husimi1940,Schleich2001}. If one
chooses to interpret a seam as
\[
s(x,p) = -\log Q(x,p) + \text{const},
\]
then \(\exp(-s)\) is normalisable and the seam parameters can be related to the
covariance of the Husimi distribution.

\paragraph{Parameters versus moments.}
The symbols \(\sigma_x,\sigma_p\) in this manuscript denote quadratic width parameters
used to normalise coordinates \(u=x/\sigma_x\), \(v=p/\sigma_p\). For non-Gaussian seams,
the true moments computed from the model density \(\propto\exp(-s_4)\) satisfy
\(\Delta x\neq \sigma_x\) and \(\Delta p\neq\sigma_p\) in general; an explicit
perturbative computation is given in Section~\ref{subsec:true-variances}.

\paragraph{Uncertainty as motivation.}
For Gaussian states, the Robertson--Schr\"odinger inequality
\cite{Robertson1929,Schrodinger1930} can be re-expressed as a minimal-area statement for
the corresponding covariance ellipse, which coincides with a quadratic seam level set.
For general non-Gaussian seams, there is no axiom of quantum mechanics that directly
constrains the Lebesgue area of an arbitrary Taylor-level set \(\{s\le 1/2\}\). When we
use a quantum-motivated area floor (Assumption~\ref{assump:area}) below, it should be
read as a modelling hypothesis. For symplectic-geometric context, one may compare with
de Gosson's \emph{quantum blobs}: minimal phase-space ellipsoids compatible with uncertainty (up to
conventions) \cite{deGosson2006}.

\paragraph{Relation to information geometry.}
Defining a Riemannian metric from a log-density is closely related to the Fisher
information metric \cite{Rao1945,AmariNagaoka2000}. Our metric is the pointwise Hessian
of \(s\); taking expectations of this Hessian under \(\exp(-s)\) yields the usual Fisher
information matrix for location parameters.

%------------------------------------------------------------------
\section{The Phase-Space Seam}
%------------------------------------------------------------------

Let \(\mathbb{R}^{2d}\) be equipped with coordinates
\((x,p) = (x_1,\dots,x_d,p_1,\dots,p_d)\). For fixed positive parameters
\(\sigma_x = (\sigma_{x_1},\dots,\sigma_{x_d})\) and
\(\sigma_p = (\sigma_{p_1},\dots,\sigma_{p_d})\) we define the \emph{phase-space seam} by
\begin{equation}
s(x,p) := \sum_{i=1}^d \left(
  \frac{x_i^2}{2\sigma_{x_i}^2} + \frac{p_i^2}{2\sigma_{p_i}^2}
\right).
\label{eq:seam-def}
\end{equation}
When all widths are equal (\(\sigma_{x_i}\equiv\sigma_x\),
\(\sigma_{p_i}\equiv\sigma_p\)) this reduces to the isotropic form
\[
s(x,p) = \frac{|x|^2}{2\sigma_x^2} + \frac{|p|^2}{2\sigma_p^2}.
\]
In the geometric development below, \(\sigma_{x_i}\) and \(\sigma_{p_i}\) are treated as
positive scale parameters. When one chooses to identify the quadratic seam with the
negative log-density of a Gaussian quantum phase-space distribution, these parameters
can be matched to physical standard deviations, in which case they obey the Heisenberg
uncertainty relation
\begin{equation}
\sigma_{x_i} \sigma_{p_i} \ge \frac{\hbar}{2} \qquad \text{for each } i=1,\dots,d.
\label{eq:heisenberg-constraint}
\end{equation}

%------------------------------------------------------------------
\section{The Hessian Rule and Induced Geometry}
%------------------------------------------------------------------

\begin{definition}[Hessian Rule]
Given a twice-differentiable seam \(s:\mathbb{R}^{2d}\to\mathbb{R}\), the \emph{Hessian
rule} assigns to \(s\) the symmetric bilinear form given by its second derivatives,
\[
g_{ij} = \frac{\partial^2 s}{\partial q^i \partial q^j}, \qquad q = (x,p).
\]
If \(g\) is positive definite it defines a Riemannian metric on \(\mathbb{R}^{2d}\).
\end{definition}

\begin{remark}[Relation to Fisher information geometry]
If one interprets \(s\) as a negative log-density, then the Hessian \(\partial_i\partial_j s\)
is the \emph{observed information} matrix. Taking the expectation of this Hessian under the
corresponding density yields the Fisher information matrix for location parameters under mild
regularity conditions. This places the Hessian rule in close proximity to standard information
geometry \cite{Rao1945,AmariNagaoka2000}.
\end{remark}

Applying the Hessian rule to the phase-space seam \eqref{eq:seam-def} yields the
constant diagonal metric
\begin{equation}
g = \operatorname{diag}\Bigl(
  \tfrac{1}{\sigma_{x_1}^2},\dots,\tfrac{1}{\sigma_{x_d}^2},
  \tfrac{1}{\sigma_{p_1}^2},\dots,\tfrac{1}{\sigma_{p_d}^2}
\Bigr).
\label{eq:hessian-metric}
\end{equation}
Its inverse is likewise diagonal with entries \(\sigma_{x_i}^2\) and \(\sigma_{p_i}^2\).
Because the second derivatives are constant, the Christoffel symbols vanish identically
and the Riemannian curvature tensor of \(g\) is zero everywhere. The geometry is
therefore flat, yet non-Euclidean: it is a simple rescaling of the standard Euclidean
metric with different stretch factors along the position and momentum axes.

The level sets \(\{ (x,p) \mid s(x,p) = c \}\) for \(c>0\) are ellipsoids. In the
isotropic case the Lebesgue area (or \(2d\)-volume) enclosed by the level set \(s=1/2\) is
\[
A = \frac{\pi^d}{d!}\prod_{i=1}^d \sigma_{x_i}\sigma_{p_i}.
\]
Imposing the scale constraint \eqref{eq:heisenberg-constraint} on each pair
therefore forces
\begin{equation}
A \ge \frac{\pi^d}{d!} \Bigl(\frac{\hbar}{2}\Bigr)^d,
\label{eq:minimal-area}
\end{equation}
with equality if and only if \(\sigma_{x_i}\sigma_{p_i} = \hbar/2\) for all \(i\).

%------------------------------------------------------------------
\section{A Minimal-Area Constraint for Quadratic Seams}
%------------------------------------------------------------------

\begin{theorem}[Quadratic sublevel-set volume under a scale constraint]
Let \(s\) be any phase-space seam of the quadratic form \eqref{eq:seam-def}. The
Lebesgue volume of the sublevel set \(\{s\le 1/2\}\) satisfies
\[
\operatorname{Vol}_{\mathrm{Leb}}\bigl(\{s\le 1/2\}\bigr) \ge \frac{\pi^d}{d!}\Bigl(\frac{\hbar}{2}\Bigr)^d,
\]
with equality when equality holds in \eqref{eq:heisenberg-constraint} for every pair.
\end{theorem}

\begin{proof}
The sublevel set \(\{s\le 1/2\}\) is the ellipsoid
\[
\sum_{i=1}^d \Bigl(
  \frac{x_i^2}{\sigma_{x_i}^2} + \frac{p_i^2}{\sigma_{p_i}^2}
\Bigr) \le 1.
\]
The Lebesgue volume of a \(2d\)-dimensional ellipsoid with semi-axes
\(\sigma_{x_i}\) and \(\sigma_{p_i}\) is \(\frac{\pi^d}{d!}\prod_i \sigma_{x_i}\sigma_{p_i}\).
The scale constraint \(\sigma_{x_i}\sigma_{p_i}\ge\hbar/2\) then gives
\[
\operatorname{Vol}_{\mathrm{Leb}}\bigl(\{s\le 1/2\}\bigr) = \frac{\pi^d}{d!}\prod_i\sigma_{x_i}\sigma_{p_i} \ge \frac{\pi^d}{d!}\Bigl(\frac{\hbar}{2}\Bigr)^d,
\]
with saturation when equality holds in every pair. (The Riemannian volume with respect to
the Hessian metric \eqref{eq:hessian-metric} is the constant \(\pi^d/d!\), independent
of the widths.)
\end{proof}

Thus a minimal-area bound for quadratic seam sublevel sets follows directly from the
volume formula for ellipsoids together with the scale constraint
\eqref{eq:heisenberg-constraint}. In quantum-mechanical applications this constraint is
motivated by Robertson--Schr\"odinger in the Gaussian case; for non-Gaussian seams we will
state any such area requirement explicitly as an assumption.

%------------------------------------------------------------------
\section{Immediate Computational Payoffs}
%------------------------------------------------------------------

The phase-space seam formulation immediately yields two practical tools that are not
obvious from a purely covariance-matrix description.

\paragraph{Inverse seam design via convex optimisation.}
Given a target covariance structure (i.e., desired pairs \((\sigma_{x_i},\sigma_{p_i})\)
or a full target ellipsoid), one can solve for the quadratic coefficients of \(s\) by
minimising a strictly convex quadratic program whose variables are the diagonal entries
of the Hessian. The scale constraint \(\sigma_x \sigma_p \ge \hbar/2\) appears as
a set of simple quadratic inequalities. The resulting seam can be used to generate
families of width-constrained seams or to enforce area/width bounds in variational
calculations.

\paragraph{Visual and numerical diagnostics.}
The seam surface \(s(x,p)\) and its Hessian eigenvalues provide an immediate visual
diagnostic: colour the phase-space domain by the product of the two eigenvalues of the
local \(2\times2\) blocks of \(\operatorname{Hess} s\). Regions where the product
approaches \(\hbar/2\) are visually ``saturated'' relative to the chosen scale.

%------------------------------------------------------------------
\section{Higher-Order Area Corrections from the Full 4-Jet}
\label{sec:fourjet}
%------------------------------------------------------------------

The quadratic seam retains only the 2-jet of a general smooth seam. We now show that
retaining the \emph{full} 4-jet --- including the cross-coupling between position and
momentum directions --- leads to a first-order correction to sublevel-set area whose
structure is richer than the diagonal treatment alone reveals. We restrict to dimension \(d = 1\)
for explicit calculations; the extension to \(d > 1\) is discussed in
Section~\ref{sec:higher-d}.

%------------------------------------------------------------------
\subsection{The Full Quartic Seam and Its Cross-Coupling}
\label{subsec:full-quartic}
%------------------------------------------------------------------

The full quartic Taylor expansion of a smooth seam \(s\) in the normalised
coordinates \(u = x/\sigma_x\), \(v = p/\sigma_p\) takes the form
\begin{equation}
s_4(u,v) = \frac{u^2 + v^2}{2}
  + \frac{\lambda}{4}\bigl(a u^4 + b v^4 + 2c\, u^2 v^2\bigr),
\label{eq:quartic-full}
\end{equation}
where \(a, b \ge 0\) and \(c \in \mathbb{R}\) are dimensionless shape parameters and
\(\lambda > 0\) sets the overall scale of the quartic correction. Restoring physical
units,
\begin{equation}
s_4(x,p) = \frac{x^2}{2\sigma_x^2} + \frac{p^2}{2\sigma_p^2}
  + \frac{\lambda}{4}\!\left(
    a\,\frac{x^4}{\sigma_x^4}
    + b\,\frac{p^4}{\sigma_p^4}
    + 2c\,\frac{x^2 p^2}{\sigma_x^2 \sigma_p^2}
  \right).
\label{eq:quartic-seam-full}
\end{equation}
The quadratic seam studied in earlier work corresponds to the restriction \(a = b = 1\),
\(c = 0\), with \(\lambda = \lambda_{\text{diag}}\). The cross-coupling coefficient \(c\)
controls the extent to which fourth-order fluctuations in \(x\) and \(p\) are
correlated.

%------------------------------------------------------------------
\subsection{True Second Moments of \texorpdfstring{$\exp(-s_4)$}{exp(-s4)} Versus the Parameters}
\label{subsec:true-variances}
%------------------------------------------------------------------

An important interpretational point is that for non-Gaussian seams, the parameters
\(\sigma_x\) and \(\sigma_p\) used in \eqref{eq:quartic-seam-full} are \emph{not} equal to
the true standard deviations computed from the model density
\(\rho(x,p) \propto \exp(-s_4(x,p))\).

To make this explicit, work in normalised variables \(u=x/\sigma_x\), \(v=p/\sigma_p\)
and consider the probability density on \(\mathbb{R}^2\)
\[
\rho_\lambda(u,v) := Z_\lambda^{-1}\,\exp\!\left(-\frac{u^2+v^2}{2} - \frac{\lambda}{4}
\left(a u^4 + b v^4 + 2c u^2v^2\right)\right),
\]
where \(Z_\lambda\) is the normalising constant.

\begin{proposition}[First-order variances of the quartic model density]
\label{prop:true-variances}
Assume \(a,b\ge 0\), \(c\ge -\sqrt{ab}\), and \(\lambda\) is sufficiently small that the
perturbation expansion is valid. Let expectations \(\langle\cdot\rangle_\lambda\) be taken
with respect to \(\rho_\lambda\). Then, to first order in \(\lambda\),
\begin{align}
\langle u^2\rangle_\lambda &= 1 - \lambda(3a + c) + O(\lambda^2),
\label{eq:u2-moment}\\
\langle v^2\rangle_\lambda &= 1 - \lambda(3b + c) + O(\lambda^2).
\label{eq:v2-moment}
\end{align}
Equivalently, the physical second moments of \(\rho(x,p)\propto \exp(-s_4(x,p))\) are
\begin{align}
\Delta x^2_{\rho} &= \langle x^2\rangle - \langle x\rangle^2
  = \sigma_x^2\bigl(1 - \lambda(3a+c) + O(\lambda^2)\bigr),
\label{eq:dx2}\\
\Delta p^2_{\rho} &= \langle p^2\rangle - \langle p\rangle^2
  = \sigma_p^2\bigl(1 - \lambda(3b+c) + O(\lambda^2)\bigr).
\label{eq:dp2}
\end{align}
\end{proposition}

\begin{proof}
Write \(Q(u,v)=a u^4 + b v^4 + 2c u^2v^2\) and denote expectations under the standard
Gaussian density \(\propto\exp(-(u^2+v^2)/2)\) by \(\langle\cdot\rangle_0\). For any test
function \(f\), the standard perturbation identity gives
\[
\langle f\rangle_\lambda
= \frac{\langle f\,e^{-\lambda Q/4}\rangle_0}{\langle e^{-\lambda Q/4}\rangle_0}
= \langle f\rangle_0 - \frac{\lambda}{4}\bigl(\langle fQ\rangle_0 - \langle f\rangle_0\langle Q\rangle_0\bigr)
+ O(\lambda^2).
\]
Using Gaussian moments \(\langle u^2\rangle_0=\langle v^2\rangle_0=1\),
\(\langle u^4\rangle_0=\langle v^4\rangle_0=3\), \(\langle u^6\rangle_0=\langle v^6\rangle_0=15\), and independence
\(\langle u^{2m}v^{2n}\rangle_0=\langle u^{2m}\rangle_0\langle v^{2n}\rangle_0\), we obtain
\(\langle Q\rangle_0=3a+3b+2c\),
\(\langle u^2Q\rangle_0=15a+3b+6c\), and
\(\langle v^2Q\rangle_0=3a+15b+6c\). Substitution yields
\eqref{eq:u2-moment}--\eqref{eq:v2-moment}. Symmetry implies \(\langle u\rangle_\lambda=\langle v\rangle_\lambda=0\), so
\eqref{eq:dx2}--\eqref{eq:dp2} follow by rescaling back to \(x,p\).
\end{proof}

\begin{remark}[Implication for interpreting \texorpdfstring{$\sigma_x\sigma_p$}{sigmax sigmap}]
For \(\lambda>0\) and typical non-Gaussian parameters (e.g. \(a,b\ge 0\) and modest
\(c\)), Proposition~\ref{prop:true-variances} shows
\(\Delta x_{\rho}<\sigma_x\) and \(\Delta p_{\rho}<\sigma_p\) at first order. Therefore,
any inequality of the form \(\sigma_x\sigma_p\ge \hbar/2\,(1+\cdots)\) should be read as
a constraint on the \emph{jet parameters} \((\sigma_x,\sigma_p)\), not automatically as a
sharpened statement about either (i) the operator uncertainties \(\Delta x\Delta p\) of a
Hilbert-space state, or (ii) the second moments of the model density \(\exp(-s_4)\).
\end{remark}

\begin{remark}[Normalisability constraint]
For the density \(\exp(-s_4)\) to be normalisable, the quartic form
\(Q(u,v) = a u^4 + b v^4 + 2c\, u^2 v^2\) must be non-negative for all
\((u,v) \in \mathbb{R}^2\). Evaluating along the ray \(u = v\) gives
\(Q = (a + b + 2c)t^4\), so we need \(a + b + 2c \ge 0\). Evaluating along
real rays yields the sharp condition. For \(v\neq 0\), write \(t:=(u/v)^2\ge 0\), so
\(
Q(u,v)=v^4\,(a t^2 + 2ct + b).
\)
If \(c\ge 0\) this is clearly nonnegative for all \(t\ge 0\). If \(c<0\) and \(a>0\), the minimum over
\(t\ge 0\) occurs at \(t=-c/a\), so nonnegativity forces \(b-c^2/a\ge 0\), i.e. \(c^2\le ab\). (The degenerate
cases \(a=0\) or \(b=0\) reduce similarly.) Hence the necessary and sufficient condition is
\begin{equation}
c \ge -\sqrt{ab}.
\label{eq:normalisability}
\end{equation}
This is the \emph{normalisability constraint}. The diagonal case \(c = 0\) satisfies it
whenever \(a, b \ge 0\).
\end{remark}

%------------------------------------------------------------------
\subsection{The 4-Jet Rule and the Correct Invariant Scalar}
\label{subsec:fourjet-scalar}
%------------------------------------------------------------------

\begin{definition}[4-Jet Rule]
Given a four-times-differentiable seam \(s : \mathbb{R}^{2d} \to \mathbb{R}\), the
\emph{4-jet rule} assigns to \(s\) the totally symmetric covariant \((0,4)\)-tensor
\[
T_{ijkl} = \frac{\partial^4 s}{\partial q^i \partial q^j \partial q^k \partial q^l},
\qquad q = (x,p).
\]
\end{definition}

For the full quartic seam \eqref{eq:quartic-seam-full} the non-zero components of the
fourth-derivative tensor are
\begin{align}
T_{xxxx} &= \frac{\partial^4 s_4}{\partial x^4}
  = \frac{6\lambda\, a}{\sigma_x^4}, \label{eq:T4-diag-x}\\
T_{pppp} &= \frac{\partial^4 s_4}{\partial p^4}
  = \frac{6\lambda\, b}{\sigma_p^4}, \label{eq:T4-diag-p}\\
T_{xxpp} &= \frac{\partial^4 s_4}{\partial x^2\,\partial p^2}
  = \frac{2\lambda\, c}{\sigma_x^2\,\sigma_p^2}. \label{eq:T4-cross}
\end{align}
Note that \(T_{xxpp}\) counts with the combinatorial multiplicity
\(\binom{4}{2} = 6\) under full index symmetrisation (the six distinct placements of two \(x\)-indices
and two \(p\)-indices), so its contribution to a full contraction is weighted by 6.

\begin{definition}[Corrected 4-Jet Scalar]
\label{def:K4-full}
The \emph{4-jet scalar} of the full quartic seam \(s_4\) is the dimensionless quantity
formed by full contraction of \(T_{ijkl}\) with the inverse Hessian metric
\((g^{-1})^{\otimes 2}\):
\begin{equation}
K_4[s_4] := \frac{1}{6}\Bigl(
  (g^{xx})^2 T_{xxxx}
  + (g^{pp})^2 T_{pppp}
  + 6\,(g^{xx})(g^{pp})\,T_{xxpp}
\Bigr),
\label{eq:fourjet-scalar-full}
\end{equation}
where \(g^{xx} = \sigma_x^2\) and \(g^{pp} = \sigma_p^2\) are the diagonal entries of
the inverse Hessian metric, and the factor \(6\) multiplying the cross-term accounts for
the combinatorial weight of two-\(x\) two-\(p\) index arrangements.
\end{definition}

\begin{remark}[Incompleteness of the diagonal definition]
The definition used in earlier diagonal treatments sets
\(K_4^{\text{diag}} = (1/6)\bigl((g^{xx})^2 T_{xxxx} + (g^{pp})^2 T_{pppp}\bigr)\),
omitting the cross-contraction entirely. This is coordinate-invariant only within the
restricted class of seams with \(c = 0\). For a general smooth seam, the full tensor
\(T_{ijkl}\) has a non-zero \(T_{xxpp}\) component, and the correct invariant is
\eqref{eq:fourjet-scalar-full}. Setting \(c = 0\) in \eqref{eq:fourjet-scalar-full}
recovers the diagonal definition as a special case.
\end{remark}

Substituting \eqref{eq:T4-diag-x}--\eqref{eq:T4-cross} into \eqref{eq:fourjet-scalar-full}:
\begin{equation}
K_4[s_4] = \frac{1}{6}\!\left(
  \sigma_x^4 \cdot \frac{6\lambda a}{\sigma_x^4}
  + \sigma_p^4 \cdot \frac{6\lambda b}{\sigma_p^4}
  + 6\,\sigma_x^2\sigma_p^2 \cdot \frac{2\lambda c}{\sigma_x^2\sigma_p^2}
\right)
= \lambda(a + b + 2c).
\label{eq:K4-computed}
\end{equation}
The diagonal scalar satisfies \(K_4^{\text{diag}} = \lambda(a+b)\), so the full scalar
exceeds it by \(2\lambda c\), which can be positive or negative depending on the sign
of \(c\).

%------------------------------------------------------------------
\subsection{Area Reduction: The Full Quartic Case}
\label{subsec:area-full}
%------------------------------------------------------------------

\begin{proposition}[Quartic Confinement, Full Version]
\label{prop:area-reduction-full}
For any quartic seam \(s_4\) with \(a,b \ge 0\) and \(\lambda > 0\), the sublevel set
\(\{s_4 \le 1/2\}\) is strictly contained in the sublevel set \(\{s_2 \le 1/2\}\)
of the corresponding quadratic seam. Consequently,
\[
A(s_4) := \operatorname{Area}\!\bigl(\{s_4 \le \tfrac{1}{2}\}\bigr) < \pi\sigma_x\sigma_p.
\]
\end{proposition}

\begin{proof}
Since \(a,b \ge 0\) and the normalisability constraint \eqref{eq:normalisability}
gives \(c \ge -\sqrt{ab}\), the quartic form \(Q(u,v) = au^4 + bv^4 + 2cu^2v^2 \ge 0\)
for all \((u,v)\). Therefore \(s_4(u,v) \ge s_2(u,v)\) for all \((u,v)\), and the
containment \(\{s_4 \le 1/2\} \subseteq \{s_2 \le 1/2\}\) follows. Strict containment
holds for all \((u,v)\ne(0,0)\) where \(Q > 0\), which is a set of positive area.
\end{proof}

%------------------------------------------------------------------
\subsection{First-Order Area Computation with Cross-Coupling}
\label{subsec:area-computation}
%------------------------------------------------------------------

In normalised coordinates \(u = x/\sigma_x\), \(v = p/\sigma_p\), write \(z=(u,v)\in\mathbb{R}^2\)
and denote by \(\widetilde T_{ijkl} := \partial_i\partial_j\partial_k\partial_l s_4(0)\) the
quartic jet tensor in these coordinates. Since the quadratic jet is \(\tfrac12|z|^2\), we may
write the quartic seam in jet form as
\[
s_4(z)=\frac12|z|^2+\frac{1}{24}\,\widetilde T_{ijkl}\,z^i z^j z^k z^l\qquad(\widetilde T=O(\lambda)).
\]
For a unit direction \(n\in S^1\), let \(r(n)\) be the radial function of the level set
\(\{s_4=1/2\}\), i.e. \(s_4(r(n)\,n)=1/2\). Then
\[
\frac12 r(n)^2+\frac{1}{24}r(n)^4\,\widetilde T(n,n,n,n)=\frac12,
\qquad \widetilde T(n,n,n,n)=\widetilde T_{ijkl}n^i n^j n^k n^l.
\]
Solving perturbatively gives
\begin{equation}
r(n)^2=1-\frac{1}{12}\,\widetilde T(n,n,n,n)+O(\lambda^2).
\label{eq:r2-expansion-invariant}
\end{equation}
The (normalised) area is therefore
\[
A_{uv}=\frac12\int_{S^1} r(n)^2\,d\Omega
=\pi-\frac{1}{24}\int_{S^1}\widetilde T(n,n,n,n)\,d\Omega+O(\lambda^2).
\]
By \(O(2)\)-invariance of the sphere, the fourth moment has the standard form
\begin{equation}
\int_{S^1} n^i n^j n^k n^l\,d\Omega
=\frac{\pi}{4}\bigl(\delta^{ij}\delta^{kl}+\delta^{ik}\delta^{jl}+\delta^{il}\delta^{jk}\bigr),
\label{eq:sphere-4-moment-S1}
\end{equation}
so, using full symmetry of \(\widetilde T\),
\begin{equation}
\int_{S^1}\widetilde T(n,n,n,n)\,d\Omega
=\frac{3\pi}{4}\,\widetilde T_{iijj}.
\label{eq:spherical-average-double-trace}
\end{equation}
Here \(\widetilde T_{iijj}\) is the double trace (with respect to the quadratic metric \(\delta\)).
For \eqref{eq:quartic-full}, direct differentiation gives
\begin{equation}
\widetilde T_{iijj}=\widetilde T_{uuuu}+2\widetilde T_{uuvv}+\widetilde T_{vvvv}
=6\lambda(a+b)+4\lambda c
=2\lambda(3a+3b+2c).
\label{eq:double-trace-abc}
\end{equation}
Substituting \eqref{eq:spherical-average-double-trace}--\eqref{eq:double-trace-abc} yields
\(A_{uv}=\pi\bigl(1-\lambda(3a+3b+2c)/16\bigr)+O(\lambda^2)\), recovering
\eqref{eq:area-normalised} below.

Equivalently, in polar coordinates \(u=r\cos\theta\), \(v=r\sin\theta\), the boundary
\(\{s_4=1/2\}\) satisfies
\begin{equation}
\frac{r^2}{2} + \frac{\lambda}{4}\,r^4\,F(\theta) = \frac{1}{2},
\label{eq:polar-levelset-full}
\end{equation}
where
\begin{equation}
F(\theta) := a\cos^4\theta + b\sin^4\theta + 2c\cos^2\theta\sin^2\theta.
\label{eq:F-def}
\end{equation}

Solving \eqref{eq:polar-levelset-full} perturbatively by writing
\(r^2 = 1 - \varepsilon\) and expanding to first order in \(\lambda\) gives
\begin{equation}
r^2(\theta) = 1 - \frac{\lambda}{2}\,F(\theta) + O(\lambda^2).
\label{eq:r2-expansion}
\end{equation}

The three angular integrals required are standard:
\begin{align}
\int_0^{2\pi} \cos^4\theta\,d\theta &= \int_0^{2\pi}\sin^4\theta\,d\theta
  = \frac{3\pi}{4}, \label{eq:int-diag}\\
\int_0^{2\pi} \cos^2\theta\sin^2\theta\,d\theta &= \frac{\pi}{4}. \label{eq:int-cross}
\end{align}
Note that the cross-coupling integral \eqref{eq:int-cross} equals \(\pi/4\), while
each diagonal integral \eqref{eq:int-diag} equals \(3\pi/4\): the cross-angular mode
\(\cos^2\theta\sin^2\theta\) integrates to exactly one-third of the diagonal modes.
This ratio is a geometric invariant that will determine the relative weighting of
\(c\) versus \(a,b\) in the conditional width bound.

Therefore,
\begin{equation}
\int_0^{2\pi} F(\theta)\,d\theta
= \frac{3\pi a}{4} + \frac{3\pi b}{4} + 2c\cdot\frac{\pi}{4}
= \frac{\pi}{4}\bigl(3a + 3b + 2c\bigr).
\label{eq:F-integral}
\end{equation}

The area in normalised coordinates is
\begin{align}
A_{uv} &= \frac{1}{2}\int_0^{2\pi} r^2(\theta)\,d\theta \notag\\
  &= \frac{1}{2}\int_0^{2\pi}\!\left(1 - \frac{\lambda}{2}F(\theta)\right)
    d\theta + O(\lambda^2) \notag\\
  &= \pi - \frac{\lambda}{4}\cdot\frac{\pi}{4}(3a + 3b + 2c) + O(\lambda^2) \notag\\
  &= \pi\!\left(1 - \frac{\lambda(3a + 3b + 2c)}{16}\right) + O(\lambda^2).
\label{eq:area-normalised}
\end{align}

Restoring physical units:
\begin{equation}
\boxed{
A(s_4) = \pi\sigma_x\sigma_p\!\left(1 - \frac{\lambda(3a + 3b + 2c)}{16}\right)
+ O(\lambda^2).
}
\label{eq:area-quartic-full}
\end{equation}

\begin{remark}[The coefficient \(1/8\) versus \(3/16\)]
The coefficient of the cross-coupling \(c\) in the area correction is \(\lambda/8\),
while the coefficient of each diagonal term \(a\) or \(b\) is \(3\lambda/16\). Their
ratio is \(2/3\), reflecting the $2c$ prefactor in \(Q\); the underlying angular integral
ratio is \(1/3\), equal to
\(\int\cos^2\theta\sin^2\theta\,d\theta \,/\, \int\cos^4\theta\,d\theta = (\pi/4)/(3\pi/4)\).
Invariantly, this ratio is a consequence of the \(O(2)\)-invariant fourth-moment tensor
\eqref{eq:sphere-4-moment-S1}: only the double-trace component \(\widetilde T_{iijj}\) of
the quartic jet contributes to the first-order area variation, and the diagonal/cross
weights follow from contracting \(\widetilde T\) against
\(\delta^{ij}\delta^{kl}+\delta^{ik}\delta^{jl}+\delta^{il}\delta^{jk}\).

This ratio of \(1/3\) between cross and diagonal angular integrals is a universal feature
independent of \(\sigma_x\) and \(\sigma_p\): it encodes the angular geometry of the
\(u\)-\(v\) plane and would appear unchanged in any rotation-invariant rescaling of the
coordinates.
\end{remark}

%------------------------------------------------------------------
\subsection{Invariant First Variation in \texorpdfstring{$n=2d$}{n=2d} Dimensions}
\label{subsec:invariant-variation}
%------------------------------------------------------------------

The preceding derivation in \(n=2\) dimensions is a special case of a general coordinate-free
statement: to first order, only the \emph{\(O(n)\)-scalar} (double-trace) component of the quartic
jet can affect the sublevel-set volume.

\begin{theorem}[Invariant first variation of sublevel-set volume]
\label{thm:invariant-first-variation}
Let \(s:\mathbb{R}^n\to\mathbb{R}\) be \(C^4\) in a neighbourhood of the origin, with
\(s(0)=0\), \(\nabla s(0)=0\), and positive definite quadratic jet
\(g_{ij}=\partial_i\partial_j s(0)\). Let \(T_{ijkl}=\partial_i\partial_j\partial_k\partial_l s(0)\)
be the fully symmetric quartic jet tensor. For \(\varepsilon>0\) sufficiently small, define the
sublevel-set volume
\[
V(\varepsilon):=\operatorname{Vol}_{\mathrm{Leb}}\bigl(\{z\in\mathbb{R}^n: s(z)\le \varepsilon\}\bigr).
\]
Then
\begin{equation}
V(\varepsilon)
=\frac{\omega_n}{n}\,\frac{(2\varepsilon)^{n/2}}{\sqrt{\det g}}
\left(1-\frac{\varepsilon}{4(n+2)}\,\mathrm{Tr}_g\mathrm{Tr}_g(T)+O(\varepsilon^2)+O(\|T\|^2)\right),
\label{eq:invariant-first-variation}
\end{equation}
where \(\omega_n\) is the surface area of \(S^{n-1}\) and
\(\mathrm{Tr}_g\mathrm{Tr}_g(T):=g^{ij}g^{kl}T_{ijkl}\) is the double trace with respect to \(g\).
\end{theorem}

\begin{proof}
By a linear change of variables \(y=Lz\) with \(L^T g L=I\), we may assume \(g=I\) at the cost of a
Jacobian factor \(1/\sqrt{\det g}\) in Lebesgue volume; the tensor \(T\) transforms covariantly and the
contraction \(\mathrm{Tr}_g\mathrm{Tr}_g(T)\) is invariant.

With \(g=I\), write \(z=rn\) with \(n\in S^{n-1}\). The jet expansion gives
For sufficiently small \(\varepsilon>0\), the sublevel set \(\{s\le \varepsilon\}\) is star-shaped about the origin, so its boundary can be written as \(\{r(n)\,n: n\in S^{n-1}\}\) for a smooth radial function \(r(n)>0\).
\(s(rn)=\tfrac12 r^2+\tfrac1{24}r^4 T(n,n,n,n)+O(r^6)\). Solving \(s(rn)=\varepsilon\) perturbatively yields
\(r(n)^2=2\varepsilon-\tfrac13\varepsilon^2 T(n,n,n,n)+O(\varepsilon^3)+O(\|T\|^2)\). Hence
\(r(n)^n=(2\varepsilon)^{n/2}\left(1-\tfrac{n\varepsilon}{12}T(n,n,n,n)\right)+\cdots\). Using
\(V(\varepsilon)=\tfrac1n\int_{S^{n-1}} r(n)^n\,d\Omega\) gives
\[
V(\varepsilon)=\frac{\omega_n}{n}(2\varepsilon)^{n/2}-\frac{(2\varepsilon)^{n/2}}{12}\,\varepsilon\int_{S^{n-1}}T(n,n,n,n)\,d\Omega+\cdots.
\]
By \(O(n)\)-invariance, the fourth moment of the uniform measure on \(S^{n-1}\) has the universal form
\begin{equation}
\int_{S^{n-1}} n^i n^j n^k n^l\,d\Omega
= \frac{\omega_n}{n(n+2)}\bigl(\delta^{ij}\delta^{kl}+\delta^{ik}\delta^{jl}+\delta^{il}\delta^{jk}\bigr).
\label{eq:sphere-4-moment-Sn}
\end{equation}
Contracting against symmetric \(T_{ijkl}\) yields
\(\int_{S^{n-1}}T(n,n,n,n)\,d\Omega=\frac{3\omega_n}{n(n+2)}\,T_{iijj}\), and substituting back gives
\eqref{eq:invariant-first-variation} (with \(T_{iijj}=\mathrm{Tr}\,\mathrm{Tr}(T)\) in the \(g=I\) frame).
\end{proof}

\begin{remark}[Only the \texorpdfstring{$O(n)$}{O(n)}-scalar part contributes at first order]
Equation \eqref{eq:sphere-4-moment-Sn} shows that the spherical averaging operator
\(T\mapsto\int_{S^{n-1}}T(n,n,n,n)\,d\Omega\) projects the quartic jet onto its unique
\(O(n)\)-invariant scalar component. Equivalently, all trace-free irreducible components of
\(T\in\mathrm{Sym}^4(\mathbb{R}^n)\) integrate to zero at first order; the first variation depends only on
\(\mathrm{Tr}_g\mathrm{Tr}_g(T)\).
\end{remark}

\begin{corollary}[Quadratic seams are locally extremal at first order]
Under the assumptions of Theorem~\ref{thm:invariant-first-variation}, let
\(s_2(z):=\tfrac12 g_{ij}z^iz^j\) be the quadratic seam with the same quadratic jet and let
\(V_2(\varepsilon):=\operatorname{Vol}_{\mathrm{Leb}}(\{z:s_2(z)\le\varepsilon\})\). Then
\begin{equation}
\frac{V(\varepsilon)}{V_2(\varepsilon)}
=1-\frac{\varepsilon}{4(n+2)}\,\mathrm{Tr}_g\mathrm{Tr}_g(T)+O(\varepsilon^2)+O(\|T\|^2).
\label{eq:quadratic-extremal-first-order}
\end{equation}
In particular, if \(\mathrm{Tr}_g\mathrm{Tr}_g(T)>0\) then \(V(\varepsilon)<V_2(\varepsilon)\) for
all sufficiently small \(\varepsilon>0\); if \(\mathrm{Tr}_g\mathrm{Tr}_g(T)<0\) then the inequality is
reversed to first order; and if \(\mathrm{Tr}_g\mathrm{Tr}_g(T)=0\) then the first-order variation vanishes.
\end{corollary}

\begin{remark}[A simple rigidity criterion under directional positivity]
The first-order statement becomes rigid under an additional positivity hypothesis.
After transforming to normal coordinates \(g=I\), suppose the homogeneous quartic form
\(n\mapsto T(n,n,n,n)\) satisfies \(T(n,n,n,n)\ge 0\) for all \(n\in S^{n-1}\).
Then \eqref{eq:sphere-4-moment-Sn} implies
\(\mathrm{Tr}\,\mathrm{Tr}(T)\ge 0\), and if \(\mathrm{Tr}\,\mathrm{Tr}(T)=0\) then
\(\int_{S^{n-1}}T(n,n,n,n)\,d\Omega=0\) forces \(T(n,n,n,n)\equiv 0\) on \(S^{n-1}\).
Since \(T(n,n,n,n)\) is a homogeneous polynomial, vanishing on the sphere implies it vanishes for all \(n\in\mathbb{R}^n\).
By the standard polarisation identity for symmetric 4-tensors, vanishing of the associated quartic form implies \(T_{ijkl}=0\).
Thus, under directional positivity, the quadratic seam is the unique seam (to quartic order)
with vanishing first variation.
\end{remark}

%------------------------------------------------------------------
\subsection{A Conditional Width Bound with Cross-Coupling}
\label{subsec:bound-full}
%------------------------------------------------------------------

\begin{assumption}[Level-set area hypothesis]
\label{assump:area}
For applications in which one postulates a quantum-motivated phase-space scale, the
Lebesgue area of the sublevel set
\(\{s\le 1/2\}\) satisfies
\(
A(s) := \operatorname{Area}(\{s\le 1/2\}) \ge \pi\hbar/2.
\)
\end{assumption}

\begin{remark}[When the area hypothesis is rigorous]
For a one-mode Gaussian state with covariance matrix \(\Sigma\), the Robertson--Schr\"odinger
inequality implies \(\det\Sigma\ge (\hbar/2)^2\) \cite{Robertson1929,Schrodinger1930}. The
covariance ellipse \(\{z^T\Sigma^{-1}z\le 1\}\) has area \(2\pi\sqrt{\det\Sigma}\), hence at
level \(s=1/2\) the corresponding quadratic seam satisfies
\(A(s)=\pi\,\Delta x\,\Delta p\ge \pi\hbar/2\). For general non-Gaussian seams, however,
\(\{s\le 1/2\}\) need not coincide with a covariance ellipse, so Assumption~\ref{assump:area}
should be viewed as an additional modelling postulate rather than a consequence of the
canonical commutation relation.
\end{remark}

\begin{theorem}[Conditional Full 4-Jet Width Bound]
\label{thm:fourjet-bound-full}
Let \(s_4\) be the full quartic seam \eqref{eq:quartic-seam-full} with parameters
\(a,b \ge 0\), \(c \ge -\sqrt{ab}\), and \(\lambda > 0\). If the sublevel set
\(\{s_4 \le 1/2\}\) satisfies Assumption~\ref{assump:area},
then to first order in \(\lambda\),
\begin{equation}
\boxed{
\sigma_x\sigma_p \;\ge\;
\frac{\hbar/2}{\,1 - \lambda(3a + 3b + 2c)/16\,}
\;\approx\;
\frac{\hbar}{2}\!\left(1 + \frac{\lambda(3a + 3b + 2c)}{16}\right).
}
\label{eq:fourjet-bound-full}
\end{equation}
This bound is valid provided \(\lambda(3a + 3b + 2c) < 16\).
For \(c = 0\) and \(a = b = 1\) this reduces to the diagonal result
\(\sigma_x\sigma_p \ge (\hbar/2)(1 + 3\lambda/8)\).
\end{theorem}

\begin{proof}
By \eqref{eq:area-quartic-full}, the minimal-area condition reads
\[
\pi\sigma_x\sigma_p\!\left(1 - \frac{\lambda(3a + 3b + 2c)}{16}\right) \ge \frac{\pi\hbar}{2},
\]
from which \eqref{eq:fourjet-bound-full} follows by dividing both sides by
\(\pi\bigl(1 - \lambda(3a+3b+2c)/16\bigr)\), which is positive by assumption.
\end{proof}

%------------------------------------------------------------------
\section{Universality: Normalisability Forces a Positive First-Order Correction}
\label{sec:universality}
%------------------------------------------------------------------

Theorem~\ref{thm:fourjet-bound-full} shows that the sign of the first-order correction
term is determined by the sign of \(3a + 3b + 2c\). This quantity can be negative if
\(c\) is sufficiently negative, which would \emph{loosen} rather than tighten the
resulting width bound under Assumption~\ref{assump:area}. The central question is
therefore: can a normalisable quartic density \(\exp(-s_4)\) achieve
\(3a + 3b + 2c \le 0\)?

The following theorem answers this question decisively in the negative.

\begin{theorem}[Universality of a positive correction coefficient]
\label{thm:universality}
For any quartic seam \(s_4\) with \(a,b \ge 0\), \(\lambda > 0\), and
cross-coupling coefficient satisfying the normalisability constraint
\(c \ge -\sqrt{ab}\), the correction coefficient satisfies
\begin{equation}
3a + 3b + 2c > 0
\label{eq:positive-correction}
\end{equation}
provided \(Q \not\equiv 0\), i.e., provided \((a,b,c) \ne (0,0,0)\).
\end{theorem}

\begin{proof}
We consider the two exhaustive cases.

\textbf{Case 1: \(a + b = 0\).}
Since \(a,b \ge 0\), this means \(a = b = 0\). For \(Q = 2cu^2v^2 \ge 0\) to
hold for all \((u,v)\) with \(Q \not\equiv 0\), we need \(c > 0\). Then
\(3a + 3b + 2c = 2c > 0\).

\textbf{Case 2: \(a + b > 0\).}
By the AM-GM inequality,
\[
\sqrt{ab} \le \frac{a + b}{2}.
\]
The normalisability constraint gives \(c \ge -\sqrt{ab}\), hence
\[
c \ge -\sqrt{ab} \ge -\frac{a+b}{2}.
\]
Therefore
\[
3a + 3b + 2c \ge 3(a + b) + 2\cdot\!\left(-\frac{a+b}{2}\right)
= 3(a+b) - (a+b) = 2(a+b) > 0. \qedhere
\]
\end{proof}

\begin{corollary}[Non-Gaussianity increases the required width product (conditional)]
Assume Assumption~\ref{assump:area}. For any normalisable non-Gaussian seam (i.e., any
quartic seam with \((a,b,c) \ne (0,0,0)\) satisfying \eqref{eq:normalisability}), the
width product \(\sigma_x\sigma_p\) is strictly bounded above \(\hbar/2\):
\[
\sigma_x\sigma_p > \frac{\hbar}{2}.
\]
The bound is saturated --- i.e., \(3a + 3b + 2c = 0\) --- only in the unphysical
limiting case \((a,b,c) = (0,0,0)\), which reduces to a Gaussian seam.
\end{corollary}

The proof of Theorem~\ref{thm:universality} reveals that the AM-GM inequality and the
normalisability constraint \eqref{eq:normalisability} are in precise correspondence: the
minimum value of \(c\) that preserves normalisability is \(-\sqrt{ab}\), and
substituting this minimum into \(3a + 3b + 2c\) gives \(3(a+b) - 2\sqrt{ab} \ge
2(a+b) > 0\), with the second inequality being strict AM-GM. The normalisability
constraint is therefore the \emph{exact} analytic condition that prevents the
cross-coupling from cancelling the first-order correction.

%------------------------------------------------------------------
\section{Extremal Seams: Minimising the Quartic Correction}
\label{sec:extremal}
%------------------------------------------------------------------

Although the correction coefficient \(3a + 3b + 2c\) is always positive, its magnitude
depends on \(c\). For fixed diagonal non-Gaussianity \(a + b > 0\), the correction is
minimised when \(c\) is as negative as normalisability allows, i.e., at the
\emph{extremal} value \(c = -\sqrt{ab}\).

\begin{definition}[Extremal Quartic Seam]
A quartic seam \(s_4\) with \(a,b > 0\) is \emph{extremal} if the cross-coupling
saturates the normalisability bound:
\begin{equation}
c = -\sqrt{ab}.
\label{eq:extremal-c}
\end{equation}
\end{definition}

\begin{proposition}[Extremal correction coefficient]
\label{prop:extremal}
For fixed \(a + b > 0\), the correction coefficient \(3a + 3b + 2c\) is minimised over
all normalisable \(c\) at the extremal value \(c = -\sqrt{ab}\), where it equals
\begin{equation}
(3a + 3b + 2c)\big|_{c=-\sqrt{ab}}
= 3(a + b) - 2\sqrt{ab}
= (\sqrt{a} + \sqrt{b})^2 + (\sqrt{a} - \sqrt{b})^2 + (a+b) > 0.
\label{eq:extremal-correction}
\end{equation}
Equivalently, \(3a + 3b - 2\sqrt{ab} = (\sqrt{3a} - \sqrt{b/3})^2 +
(8b/3) > 0\).
\end{proposition}

\begin{proof}
Since \(c \ge -\sqrt{ab}\) and \(3a + 3b + 2c\) is increasing in \(c\), the minimum is
at \(c = -\sqrt{ab}\), giving
\(3(a+b) - 2\sqrt{ab} = a + b + 2(a + b - \sqrt{ab}) \ge a + b > 0\),
where we used \(a + b \ge 2\sqrt{ab}\) by AM-GM.
\end{proof}

The extremal seam has a natural geometric interpretation. The quartic form
\(Q = au^4 + bv^4 - 2\sqrt{ab}\,u^2v^2 = (\sqrt{a}\,u^2 - \sqrt{b}\,v^2)^2 \ge 0\)
vanishes precisely along the two lines \(v = \pm(\sqrt{a/b})^{1/2} u\) in the
normalised \((u,v)\) plane. Along these lines the quartic correction is exactly zero,
and the seam is purely quadratic. In all other directions the quartic correction is
strictly positive and confines the sublevel set.

\begin{remark}[Geometric interpretation of extremal seams]
The extremal condition \(c = -\sqrt{ab}\) means that position-squared and
momentum-squared fluctuations are as \emph{negatively correlated} as possible while
maintaining normalisability. The state has strong fourth-order fluctuations in the
diagonal directions \(x \sim \pm(\sigma_x/\sigma_p)^{1/2} p\) but is constrained near
the axes. Among all non-Gaussian seams with fixed diagonal kurtosis \(a + b\), extremal
seams are the \emph{closest to quadratic} in the sense of minimising the quartic width
penalty implicit in \eqref{eq:fourjet-bound-full} under Assumption~\ref{assump:area}.
\end{remark}

%------------------------------------------------------------------
\section{Connection to the Non-Gaussianity Witness}
\label{sec:nongaussianity}
%------------------------------------------------------------------

The scalar \(K_4^{\text{full}} = \lambda(a + b + 2c)\) defined in
\eqref{eq:K4-computed} provides a coordinate-invariant witness for non-Gaussianity, but
it is \emph{not} the quantity that appears in the conditional width bound. The bound
\eqref{eq:fourjet-bound-full} is governed by the weighted combination
\begin{equation}
\mathcal{W}_4 := \lambda\,\frac{3(a+b) + 2c}{16}
= \frac{3\lambda(a+b)}{16} + \frac{\lambda c}{8}
= \frac{3K_4^{\text{diag}}}{16} + \frac{\lambda c}{8},
\label{eq:W4-def}
\end{equation}
where \(K_4^{\text{diag}} = \lambda(a+b)\) is the diagonal 4-jet scalar. The full
4-jet scalar \(K_4^{\text{full}}\) can be recovered from \(\mathcal{W}_4\) and
\(K_4^{\text{diag}}\):
\[
K_4^{\text{full}} = K_4^{\text{diag}} + 2\lambda c
= 2K_4^{\text{full}} - K_4^{\text{diag}}
\quad\Longrightarrow\quad
\lambda c = \frac{K_4^{\text{full}} - K_4^{\text{diag}}}{2}.
\]
Substituting back,
\begin{equation}
\mathcal{W}_4 = \frac{3K_4^{\text{diag}}}{16}
  + \frac{K_4^{\text{full}} - K_4^{\text{diag}}}{16}
= \frac{K_4^{\text{full}} + 2K_4^{\text{diag}}}{16}.
\label{eq:W4-from-scalars}
\end{equation}
The bound is therefore governed by the weighted average of the full and
diagonal 4-jet scalars with weights \(1:2\), not by either scalar alone. This gives a
precise statement of how much information the diagonal treatment loses: it overestimates
the correction coefficient by a factor that depends on the relative size of
\(K_4^{\text{full}}\) and \(K_4^{\text{diag}}\). When \(c > 0\) the diagonal treatment
underestimates the correction; when \(c < 0\) it overestimates.

%------------------------------------------------------------------
\section{The Jet Hierarchy}
\label{sec:hierarchy}
%------------------------------------------------------------------

Theorem~\ref{thm:fourjet-bound-full} and Theorem~\ref{thm:universality} together
establish the first two levels of a hierarchy of width bounds (conditional on
Assumption~\ref{assump:area} at quartic order) indexed by the even jets of a smooth seam:

\paragraph{Where \texorpdfstring{$\hbar$}{hbar} enters.}
The appearance of \(\hbar\) here is not intrinsic to the jet calculus itself: it enters only through
the chosen \emph{phase-space scale} used to turn geometric area/volume statements into a width-product bound.
Concretely, at quadratic order the bound \(\sigma_x\sigma_p\ge \hbar/2\) is the quantum-motivated scale constraint
\eqref{eq:heisenberg-constraint}, and at quartic order the same scale is propagated via the level-set area
hypothesis (Assumption~\ref{assump:area}). If one prefers a purely geometric presentation, one may replace
\(\hbar/2\) everywhere in the table by an abstract constant \(\kappa>0\) (equivalently, write
\(A(s)\ge \pi\kappa\) in Assumption~\ref{assump:area}); the jet-dependent correction factors
\(1+\mathcal{W}_{2k}+\cdots\) are unchanged.

\medskip
\begin{center}
\begin{tabular}{clll}
\hline
Jet & Geometric object & Width bound & Universality \\
\hline
2 & Hessian metric \(g\) & \(\sigma_x\sigma_p \ge \hbar/2\) & Always positive \\
4 & \(T_{ijkl}\), weighted scalar \(\mathcal{W}_4\)
  & \(\sigma_x\sigma_p \ge (\hbar/2)(1 + \mathcal{W}_4 + \cdots)\)
  & Positive iff normalisable \\
\(2k\) & Symmetric \(2k\)-tensor, scalar \(\mathcal{W}_{2k}\)
  & \(\sigma_x\sigma_p \ge (\hbar/2)(1 + \mathcal{W}_{2k} + \cdots)\)
  & Conjectured positive \\
\hline
\end{tabular}
\end{center}
\medskip

At the fourth-order level the universality of positive corrections follows from the
AM-GM inequality and the normalisability constraint. We conjecture that analogous
positivity statements hold at every even order \(2k\), with the relevant physical
constraint being normalisability of \(\exp(-s_{2k})\) at that order.

%------------------------------------------------------------------
\section{Multi-Mode Extension}
\label{sec:higher-d}
%------------------------------------------------------------------

For \(d > 1\), the phase space is \(\mathbb{R}^{2d}\) with coordinates
\((x_1,\dots,x_d,p_1,\dots,p_d)\), and the quartic jet acquires many more independent components.
In normalised coordinates \(q=(u_1,\dots,u_d,v_1,\dots,v_d)\) (so the quadratic jet is \(\tfrac12|q|^2\)),
write the local 4-jet expansion
\[
s(q)=\frac12|q|^2+\frac{1}{24}\,\widetilde T_{ijkl}\,q^i q^j q^k q^l+O(|q|^6),
\]
where \(\widetilde T\) is the fully symmetric quartic jet tensor in these coordinates.

The angular integrals over the unit sphere \(S^{2d-1}\) are controlled invariantly by
the spherical fourth-moment identity \eqref{eq:sphere-4-moment-Sn}. In components, this implies
\begin{align}
\int_{S^{2d-1}} q_i^4\, d\Omega &= \frac{3\,\omega_{2d}}{2d(2d+2)},
\label{eq:int-sphere-4}\\
\int_{S^{2d-1}} q_i^2 q_j^2\, d\Omega &= \frac{\omega_{2d}}{2d(2d+2)} \quad (i \ne j),
\label{eq:int-sphere-22}
\end{align}
where \(\omega_{2d}\) is the surface area of \(S^{2d-1}\). The ratio of cross to
diagonal angular integrals is again \(1:3\), independent of dimension. This means the
first-order sublevel-set volume correction depends only on the double trace of \(\widetilde T\).

Specialising Theorem~\ref{thm:invariant-first-variation} to \(n=2d\) and the level \(\varepsilon=1/2\)
(so \((2\varepsilon)^{n/2}=1\)) yields the relative correction
\begin{equation}
\mathcal{W}_4^{(d)} := \frac{1}{8(2d+2)}\,\mathrm{Tr}\,\mathrm{Tr}(\widetilde T)
= \frac{1}{8(2d+2)}\,\widetilde T_{iijj},
\label{eq:W4-multimode}
\end{equation}
which is manifestly basis-independent. If one expands \(\widetilde T_{iijj}\) in an orthonormal basis,
the same \(1:3\) diagonal-versus-cross weighting reappears (via \eqref{eq:int-sphere-4}--\eqref{eq:int-sphere-22}),
confirming that the ratio is a spherical-geometry invariant rather than a coordinate artefact.

\begin{remark}[Positivity forces a positive \texorpdfstring{$O(n)$}{O(n)}-scalar projection]
If the quartic jet induces a nonnegative homogeneous quartic form on directions,
\(\widetilde T(n,n,n,n)\ge 0\) for all \(n\in S^{2d-1}\) (as occurs, for example, when one requires
normalisability of a model density \(\propto\exp(-s)\) with a strictly positive quartic part), then
\(\int_{S^{2d-1}}\widetilde T(n,n,n,n)\,d\Omega\ge 0\). By \eqref{eq:sphere-4-moment-Sn}, this is equivalent to
\(\widetilde T_{iijj}\ge 0\), hence \(\mathcal{W}_4^{(d)}\ge 0\). Moreover, if \(\widetilde T(n,n,n,n)\) is not
identically zero on the sphere then the inequality is strict.
\end{remark}

%------------------------------------------------------------------
\section{Saturation, Gaussians, and Further Directions}
%------------------------------------------------------------------

\begin{corollary}[Saturation at every level]
The bound \eqref{eq:fourjet-bound-full} is saturated (i.e., the correction vanishes)
if and only if \(a = b = c = 0\) and \(\sigma_x\sigma_p = \hbar/2\). These conditions
hold simultaneously only for the quadratic (Gaussian) seam, confirming that any
departure from quadraticity incurs a strict first-order correction at quartic order (under
Assumption~\ref{assump:area}).
\end{corollary}

\paragraph{Curved-space extensions.}
Equip \(\mathbb{R}^{2d}\) with a background Riemannian metric \(h\) and define the seam
\(s\) relative to \(h\). The covariant Hessian rule \(g_{ij} = \nabla_i\nabla_j s\)
incorporates curvature of the background. On positively curved backgrounds the effective
area/width floor may increase; the quartic correction \eqref{eq:W4-multimode} will
acquire additional terms from curvature tensors contracted against \(T_{ijkl}\), and the
universality theorem will require a modified normalisability constraint that couples the
quartic seam coefficients to the background Ricci tensor.

\paragraph{Universal approximation conjecture.}
Every sufficiently regular, strictly positive density on \(\mathbb{R}^{2d}\) may be
approximated (in an appropriate topology) by finite mixtures of quartic exponential
models of the form \(\propto\exp(-s_4)\). The weighted scalar \eqref{eq:W4-from-scalars}
would then provide an explicit non-Gaussianity measure for each component.

%------------------------------------------------------------------
\section{Conclusion}
%------------------------------------------------------------------

Phase-space seams provide a scalar-first, entirely geometric framework for studying
Hessian-generated geometry and sublevel-set volumes on phase space. The extension to the
full quartic seam --- including the
position-momentum cross-kurtosis term \(x^2 p^2\) --- reveals structure absent from the
diagonal-only treatment:

\begin{enumerate}
\item The correct 4-jet scalar for the conditional width bound is not \(K_4^{\text{full}} =
\lambda(a+b+2c)\) but the weighted combination \(\mathcal{W}_4 = \lambda(3a+3b+2c)/16\),
in which the cross-coupling contributes with weight \(1/3\) relative to the diagonal
terms (from the angular integrals), with the overall \(1/16\) factor rather than \(1/8\)
following from the correct evaluation
\(\int_0^{2\pi}\cos^4\theta\,d\theta = 3\pi/4\) and
\(\int_0^{2\pi}\cos^2\theta\sin^2\theta\,d\theta = \pi/4\).
The ratio \(1/3\) between the angular integrals is a universal geometric invariant, and
the full expression \((\pi/4)(3a+3b+2c)\) integrating \(F(\theta)\) captures both.

\item The normalisability constraint \(c \ge -\sqrt{ab}\) is precisely the condition that
forces \(3a+3b+2c > 0\) for all non-trivial quartic seams. The proof reduces to a single
application of the AM-GM inequality. Normalisability is therefore not merely a
technical requirement but the exact analytic principle that prevents cross-coupling from
cancelling the first-order correction.

\item Among all non-Gaussian seams with fixed diagonal kurtosis \(a+b > 0\), the
extremal seams with \(c = -\sqrt{ab}\) minimise the quartic correction. They
are characterised geometrically as seams whose quartic correction vanishes along two
specific lines in phase space and is concentrated in the transverse directions.
Equivalently, at fixed diagonal kurtosis, they furnish a canonical family of
``most Gaussian-like'' normalisable quartic seams for variational calculations.

\item Under Assumption~\ref{assump:area}, the conditional width bound takes the explicit
form \(\sigma_x\sigma_p \ge (\hbar/2)(1 + \mathcal{W}_4 + O(\lambda^2))\) with
\(\mathcal{W}_4 > 0\) for all normalisable non-Gaussian seams.
\end{enumerate}

The immediate computational payoffs (convex inverse design and visual diagnostics) are
ready for implementation using \(\mathcal{W}_4\) rather than the diagonal scalar. The
sketched multi-mode extension suggests that the \(1:3\) ratio of cross to diagonal
angular integrals is a dimension-independent geometric constant that will appear at every
level of the jet hierarchy.

\paragraph{Caveat on physical interpretation.}
For non-Gaussian seams, the width parameters \(\sigma_x,\sigma_p\) are not identical to
either (i) the true second moments of the model density \(\exp(-s_4)\) (see
Proposition~\ref{prop:true-variances}) or (ii) the operator uncertainties of an
underlying Hilbert-space state without specifying an explicit phase-space mapping (e.g.
via a Husimi \(Q\)-function). Accordingly, the quartic correction derived here should
be read as a conditional geometric statement about jet parameters under
Assumption~\ref{assump:area}, not as a general strengthening of
\(\Delta x\,\Delta p\ge\hbar/2\).

%------------------------------------------------------------------
\begin{thebibliography}{99}

\bibitem{Wigner1932}
E.~Wigner,
\emph{On the Quantum Correction For Thermodynamic Equilibrium},
Phys. Rev. \textbf{40} (1932), 749--759.

\bibitem{Husimi1940}
K.~Husimi,
\emph{Some Formal Properties of the Density Matrix},
Proc. Phys.-Math. Soc. Japan \textbf{22} (1940), 264--314.

\bibitem{Hillery1984}
M.~Hillery, R.~F. O'Connell, M.~O. Scully, and E.~P. Wigner,
\emph{Distribution functions in physics: Fundamentals},
Phys. Rep. \textbf{106} (1984), 121--167.

\bibitem{Schleich2001}
W.~P. Schleich,
\emph{Quantum Optics in Phase Space},
Wiley-VCH, 2001.

\bibitem{Robertson1929}
H.~P. Robertson,
\emph{The Uncertainty Principle},
Phys. Rev. \textbf{34} (1929), 163--164.

\bibitem{Schrodinger1930}
E.~Schr\"odinger,
\emph{Zum Heisenbergschen Unsch\"arfeprinzip},
Sitzungsberichte der Preu\ss ischen Akademie der Wissenschaften, Phys.-Math. Klasse
(1930), 296--303.

\bibitem{Rao1945}
C.~R. Rao,
\emph{Information and the Accuracy Attainable in the Estimation of Statistical Parameters},
Bull. Calcutta Math. Soc. \textbf{37} (1945), 81--91.

\bibitem{AmariNagaoka2000}
S.-I. Amari and H.~Nagaoka,
\emph{Methods of Information Geometry},
Translations of Mathematical Monographs, Vol. 191, AMS, 2000.

\bibitem{Shima2007}
H.~Shima,
\emph{The Geometry of Hessian Structures},
World Scientific, 2007.

\bibitem{deGosson2006}
M.~de Gosson,
\emph{Symplectic Geometry and Quantum Mechanics},
Birkh\"auser, 2006.

\end{thebibliography}

\end{document}